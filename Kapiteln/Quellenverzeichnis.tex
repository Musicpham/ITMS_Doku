%!TEX root = ../Thesis.tex
\section*{Quellenverzeichnis}
\addcontentsline{toc}{section}{Quellenverzeichnis}
\fancyhead[R]{Quellenverzeichnis}

\defbibheading{mono}{\subsection*{Monographien}}
\defbibheading{mag}{\subsection*{Aufsätze in Sammelbänden und Zeitschriften}}
\defbibheading{art}{\subsection*{Zeitungsartikel}}
\defbibheading{web}{\subsection*{Internetquellen}}
\defbibheading{leg}{\subsection*{Rechtsprechung}}
\defbibheading{comp}{\subsection*{Unternehmensunterlagen/Gesprächsnotizen}}
\defbibheading{webinar}{\subsection*{Webinare und Seminarunterlagen}}
\defbibheading{sap}{\subsection*{SAP Hilfe}}
\defbibheading{manus}{\subsection*{Firmeninterne Manuskripte}}
\defbibheading{interview}{\subsection*{Gesprächszusammenfassungen}}

\setlength\bibitemsep{1.5\itemsep}
\setlength{\bibhang}{2em}

\renewcommand{\baselinestretch}{1.50}\normalsize

\begingroup
\sloppy

\printbibliography[keyword={mono},heading=mono]
% \printbibliography[keyword={sap},heading=sap]
% \printbibliography[keyword={mag},heading=mag]
\printbibliography[heading=web,keyword={web}]
% \printbibliography[heading=webinar,keyword={webinar}]
% \printbibliography[heading=manus,keyword={manus}]
% \printbibliography[heading=interview,keyword={interview}]
% Bei Bedarf einkommentieren: (erzeugt sonst Warnungen)
% \printbibliography[heading=art,keyword=art]
% \printbibliography[heading=leg,keyword=leg]
\printbibliography[heading=comp,keyword={comp}]

\endgroup