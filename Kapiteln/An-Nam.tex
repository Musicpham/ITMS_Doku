%!TEX root = ../Thesis.tex
\section{\glqq Was soll in die Cloud?\grqq~(An-Nam Pham)}
In Kapitel \ref{sec:IT-Infrastruktur} wurden die Vorteile für einen Umzug in die Cloud erläutert.
Da ein Teil der gesamten IT-Systemlandschaft (die zentrale Datenbank) innerhalb der Firma selbst betrieben wird, handelt es sich in diesem Fall um eine hybride Cloud, die aus der Private Cloud (zentrale Datenbank) und der Public Cloud besteht.\\
\\
Dieses Kapitel stellt unsere Empfehlung dar, was für Systeme (z.B. Module, Komponenten, etc.) in die Public Cloud installiert werden soll und was für ein Nutzen diese Systeme für das Unternehmen Stylez hat.
Die folgende Abbildung zeigt die empfohlene Implementierung der gesamten IT-Systemlandschaft. Die einzelnen Komponenten werden in den nächsten Unterkapiteln erläutert:
\begin{figure}[H]
\centering
\begin{minipage}[t]{0.8\textwidth}
\fbox{\includegraphics[width=1\textwidth]{img/Cloudinhalt.pdf}}
\caption{Empfohlene Implementierung der Hybrid Cloud} % Überschrift
\source{Eigene Darstellung} % Quelle
\label{img:Cloud_Implementierung}
\end{minipage}
\end{figure}
\subsection{SAP IT-System}
Wir empfehlen die gesamte Unternehmensstruktur von Stylez im IT-System abzubilden.
%SAP ERP (immer groß eine Überschrift mit "Der Nutzen")
%SAP CRM
%SAP Solution Manager
%SAP Rapid Deployment Solution
\subsection{Webshop}
\section{So funktioniert der Umzug in die Hybrid Cloud (An-Nam Pham)}
% \section{Stichwort: Hochverfügbarkeit}
\section{Roadmap (An-Nam Pham)}
