%!TEX root = ../Thesis.tex
\section{Einleitung (Franziska Plate)}

Im Rahmen der IT Management und Strategie-Vorlesung bei Herrn Benno
Klaas, bekam die Gruppe um An-Nam Pham, Patrick Künzl, Vasilij
Schneidermann und Franziska Plate die Aufgabe ein Beratungsprojekt für
ein expandierendes Start-up Unternehmen, namens ,,Stylez'',
durchzuführen. Hierfür bekam die Gruppe von dem ,Kunden', Herrn Klaas,
ein Briefing, welches den Hintergrund und den aktuellen Status des
Unternehmens beinhaltet.

An Hand und mit Hilfe folgender Informationen führt die Gruppe nun ein
Beratungsprojekt für das Unternehmen ,,Stylez'' durch:

Das Unternehmen ,,Stylez'' ist ein Start-up mit Sitz in Bergisch
Gladbach. Es ist seit 2 Jahren am Markt und hat bereits zwanzig
Franchise-Filialen. Geschäftsmodell von ,,Stylez'' ist es, qualitativ
hochwertige Mode in Franchise-Live-Style-Shops, sowie auf einer
Internet-Plattform anzubieten. So können die Kunden sowohl im Laden,
als auch online einkaufen. Ebenfalls haben sie die Wahl, die Ware per
Post zu erhalten oder sie in eine Filiale senden zu lassen. Außerdem
gehört es zum Konzept, dass die Franchise-Nehmer die IT Leistungen
(Internet-Auftritte inkl.~Web-Shop und Support) von der Zentrale
erhalten können. ,,Stylez'' plant, durch den großen Erfolg, bis Ende
2017 nach Österreich, in die Schweiz, Benelux-Länder, England und
Frankreich zu expandieren.

Momentan beschäftigt ,,Stylez'' in seiner Zentrale 120 Mitarbeiter,
die meisten davon im Einkauf und Marketing. Im Außendienst befinden
sich momentan zwanzig Mitarbeiter, welche die Franchise-Nehmer
betreuen und neue werben. Dadurch, dass durch den schnellen Wachstum
kein nachhaltiges IT-Konzept existiert, beschweren sich die
Mitarbeiter immer lauter darüber, dass die IT-Ausstattung nicht dem
notwendigen Standard entspräche und viele Vorgänge noch manuell
durchgeführt werden müssen. Dies führt zum einen zu hohen
Prozesskosten und die Motivation der Mitarbeiter sinkt.

Außerdem wurden folgende Probleme bereits von Mitarbeitern gemeldet:

\begin{itemize}
\item Die Arbeitsplatzausstattung, wie Rechner, Monitore und Drucker,
  ist eine Mischung aus verschiedenen Herstellern und Modellen.
\item Die eingesetzte Software ist zum Teil nicht lizenziert und wurde
  aus privaten oder aus anderen fragwürdigen Quellen beschafft.
\item Der Außendienst nutzt fünf verschiedene Arten von Laptops, zum
  Teil werden auch private Geräte verwendet.
\item Die Franchise-Filialen sind über einen normalen DSL-
  bzw. Kabel-Anschluss angebunden.
\item Die IT-Abteilung betreibt eigene Server verschiedener Hersteller
  in einem Serverschrank, welcher sich im Keller befindet. Über diesen
  Server werden alle Dienste zur Verfügung gestellt inklusive Mail und
  Hosting.
\item Es gibt keine zentrale Anlaufstelle für Support-Anfragen. Jeder
  IT-Mitarbeiter erhält jede Support-Anfrage, was dazu führt, dass
  Anfragen lange liegen bleiben und die IT-Mitarbeiter, bestehend aus
  sechs Personen, zum Teil überfordert sind.
\item Die Unternehmensdaten liegen auf einem zentralen File-Server und
werden über FTP angesprochen.
\item Der Standpunkt der Geschäftsführung ist, dass IT funktionieren
  soll und möglichst wenig Kosten verursachen
  soll\footnote{Vgl.~\cite{ITMS-Briefing}}.
\end{itemize}

Auf Basis dieses Briefings wird zu Beginn dieser Ausarbeitung eine
Problemanalyse durchgeführt und schon ein Soll-Zustand ermittelt und
erste Lösungsansätze beschrieben, welche dann im weiteren Verlauf
dieser Arbeit genauer erklärt und erläutert werden. Zum Abschluss
dieser Ausarbeitung wird noch ein abschließendes Fazit gezogen.
