\section{Weitere IT-Maßnahmen (Vasilij Schneidermann)}

In der Analyse wurden zusätzlich zu der großen Baustelle Cloud noch
drei weitere Verbesserungspotenziale identifiziert: Die uneinheitliche
Arbeitsplatzausstattung, der überforderte IT-Support und die
Optimierung der Geschäftsprozesse.  Nahezu alle dieser Bereiche werden
durch den \emph{SAP Solution Manager} gelöst, jedoch ist es teilweise
möglich andere Lösungen für eine bessere Abdeckung hinzuzuziehen.  Zu
diesen gehören das \emph{SAP IT Infrastructure Management} und
\emph{Service Desk}.

\subsection{Arbeitsplatzausstattung}

Ehe eine einheitliche Arbeitsplatzausstattung verwendet werden kann,
ist es notwendig sich auf einen, oder besser, zwei IT-Ausstatter
festzulegen welche die nötige Infrastruktur zur Anbindung der
Cloud-Services bereitstellen können.  Gründe für mehr als einen
Ausstatter sind:

\begin{itemize}
\item Mehr Freiheit bei der Auswahl, da wenige Anbieter alle nötigen
  Bereiche gut bedienen können
\item Mehr Unabhängigkeit falls ein Anbieter nicht mehr die
  gewünschten Leistungen erbringen kann
\item Eine bessere Absicherung gegen Ausnutzung einer Einzelstellung
  des Anbieters
\end{itemize}

Eine denkbare Kombination wäre z.B.~Lenovo für Business-Laptops und
Dell für die verbleibende Peripherie (Monitore, Tastaturen, Mäuse,
etc.).  Je nach örtlicher Verfügbarkeit und Konditionen können
beliebige andere Anbieter ausgesucht werden.

Nach Vereinbarung eines Kooperationsvertrages kann mit einem
schrittweisen Rollout der neuen Hardware begonnen werden.  Zwar wäre
es auch möglich die Infrastruktur auf einen Schlag zu ersetzen, jedoch
ist dies generell nicht zu empfehlen, da etwaige Probleme beim Wechsel
sich über die gesamte IT-Landschaft erstrecken würden und in diesem
Falle es zu einem Totalausfall der IT kommen kann.  Im Gegensatz dazu
ist die kontinuierliche Ersetzung der alten Hardware weniger riskant,
verteilt die eingesetzten Kosten gleichmäßig über die Zeitdauer des
Rollouts und erlaubt es über den Vorgang hinweg wertvolles Feedback
der Mitarbeiter zu sammeln und umzusetzen.

An dieser Stelle ist es sinnvoll festzuhalten welche Hardware wo
eingesetzt wird.  Zu diesem Zweck kann der \emph{SAP Solution Manager}
durch das \emph{SAP IT Infrastructure
  Management}\footnote{Vgl.~\cite{SAP-IT-Realtech}, Online im
  Internet} erweitert werden, direkt nach Verkabelung der Geräte
werden ihre Seriennummern festgehalten und in die
CMDB\footnote{Configuration Management Database} eingetragen werden.
Neue Software wird von nun an nur noch aus autorisierten Quellen nach
Anfrage aufgespielt, so wird die Lizenzproblematik gekonnt umschifft.

Von diesem Zeitpunkt an ist die Hardware Teil des ERP-Prozesses: Der
Erwerb wird ordnungsgemäß festgehalten, ausgefallene Hardware kann
angefordert und nachbestellt werden, sie ist Teil des Inventars, die
dafür notwendige Software kann mit Leichtigkeit angefordert und
lizensiert werden, mithilfe von Monitoring und Berichten ist der
Infrastrukturstatus dem höheren Management stets bekannt, etc.  Da
jetzt auch der \emph{SAP Solution Manager} über das Inventar
informiert ist, kann man die Infrastruktur nun wie jeden anderen Teil
des Unternehmens verwalten und auftretende Probleme schnell
lösen\footnote{Vgl.~\cite{SAP-IT-Infrastructure-Management}, Online im
  Internet}.

\subsection{IT-Support}

Für die Bereitstellung eines idealen IT-Supports ist es notwendig
genau einen zentralen Zugriffspunkt zu haben.  Dieser muss nicht auf
ein Kommunikationsmedium zum Einreichen der sogenannten
\emph{Incidents} beschränkt sein, denkbar ist z.B.~eine Kombination
aus Benachrichtigung vom Produkt, bzw.~Anwendung heraus sowie einem
klassischen per Telefon und E-Mail erreichbaren \emph{Service
  Desk}\footnote{Vgl.~\cite{SAP-Service-Desk}, Online im Internet}.
Diese erhalten/erfassen für jeden Incident eine einheitlich gestaltete
Problembeschreibung, Priorität sowie weitere systemrelevante Daten wie
die Version der eingesetzten Software und das Betriebssystem des Kunden.

Zu diesem Zeitpunkt kann die Supportanfrage von einem automatisierten
System auf Merkmale wie den Supportlevel und einen ggf.~schon
vorhandenen Servicevertrag analysiert werden.  Sind diese bekannt,
weist das System die Anfrage einem Mitarbeiter zu, benachrichtigt
diesen und leitet sie bei Bedarf (wie z.B.~nach einer manuellen
Zuweisung an einen anderen Mitarbeiter) weiter.  Bei Bedarf können
noch fehlende Informationen wie E-Mail-Adresse und Telefonnummer vom
Kunden angefragt werden.

Die so vervollständigte Supportanfrage wird auf eine mögliche Lösung
untersucht.  Da ein Großteil auftretender Probleme dem Support schon
bekannt und wiederkehrender Natur ist, werden die beschriebenen
Symptome in einer sogenannten \emph{Customer Solution Database}
gesucht welche mit der Zeit durch neue Anfragen erweitert und bekannte
Anfragen verbessert wird.  Findet man eine möglicherweise anwendbare
Lösung, geht es an die Umsetzung seitens des Kunden, ansonsten wird
die Anfrage an ein höheres Supportlevel delegiert.  SAP bietet dafür
ein \emph{Support Back Office} an in welchem man sich nach Anforderung
der dafür nötigen Credentials mit ihnen austauschen kann.  Die so
erhaltene Lösung wird ähnlich der Lösungsdatenbank bereitgestellt,
sodass man auf den gleichen Workflow wie bei der internen
Lösungsfindung kommt.

Als nächstes kontaktiert man den Kunden mit der gefundenen Lösung und
hilft diesem sie anzuwenden und den Erfolg festzustellen.  Ist diese
erfolgreich, hinterlegt man sie in der \emph{Customer Solution
  Database}, andernfalls iteriert man durch andere gefundene Lösungen.
Schlussendlich markiert man das Supportticket als gelöst und bittet um
Kundenfeedback.  Dieses ist extremst wichtig um den IT-Support wirksam
zu steuern und den Kunden anzupassen.  Nur so kann sichergestellt
werden, dass dieser wirkungsvoll bleibt und den Ansprüchen der Kunden
gerecht wird.

Zur Integration in das Reporting werden in dieser IT-Support-Lösung
die Ticketzahlen nach Zeit/Abteilung, die Dauer der Bearbeitung,
Kundenzufriedenheit und interessante Verteilungen protokolliert.  Zu
diesen gehören:

\begin{itemize}
\item Über die \emph{Customer Solution Database} gelöste Tickets
\item Mit externen Hilfe gelösten Tickets
\item Menge an Tickets pro Abteilung
\item Tickets mit Konsequenzen im Change Management
\end{itemize}

Eine weitere wichtige Berücksichtigung ist, dass ein bestimmter Anteil
an Supportanfragen nicht immer nur durch interne und
SAP-Beratungsstellen gelöst werden kann, sondern es bei Bedarf auch
externe Help Desks geben kann welche z.B.~durch Outsourcing oder
anderweitig spezialisierte Expertise dafür besser geeignet sind.  Das
Service Desk kann auch bei Einbindung dieser externen Ressourcen
weiterhin deren Effizienz und Effektivität festhalten und diese somit
ähnlich wie die anderen Beratungsstellen integrieren kann.

\subsection{Kennzahlen und Benchmarking}

Ein an das Unternehmen angepasstes SAP-ERP-System erfasst
verschiedenste Daten, jedoch ist es notwendig die für die Zielsetzung
und -anpassung wichtigen Kriterien gesondert zu beobachten um anhand
der Soll- und Ist-Ergebnisse die Geschäftsprozesse anpassen zu
können.  Die so gewählten Kennzahlen dürfen dabei nicht in Isolation,
sondern müssen stets im Zusammenhang mit einer Zielgröße betrachtet
werden, erst dann ist ein Benchmarking durch einen Vergleich mit einem
anderen Report, einem anderen Unternehmens oder einer anderen Branche
möglich.

SAP schlägt für IT-Unternehmen vor die Bereiche \emph{Continuity},
\emph{Efficiency} und \emph{Compliance} zu
betrachten\footnote{Vgl.~\cite{SAP-Standard-IT-Service-Agreement},
  Online im Internet}.  Für die folgende Auflistung wurden für
sinnvoll erachtete Kennzahlen aus ihren Dokumenten sowie selbst
auserwählte mit typischen Zielgrößen genommen.

\begin{itemize}
\item Continuity
  \begin{itemize}
  \item Durchschnittliche Zeit zur Erstellung einer Supportanfrage (< 30min)
  \item Durchschnittliche Zeit für administrative Aufgabe (< 15min)
  \item Prozentualer Anteil durch das Help Desk gelöster Anfragen (> 90\%)
  \item Häufigkeit der Aktualisierung der Support Database (> 2x täglich)
  \item Durchschnittliche Uptime von IT-Services (> 99,9\%)
  \end{itemize}
\item Efficiency
  \begin{itemize}
  \item Kundenzufriedenheit (> 90\%)
  \item Dauer zur Bearbeitung interner E-Mails (< 1/2 Tag)
  \item Durchschnittliche Zeit zur Bearbeitung einer IT-Anfrage (< 1 Tag)
  \item Prozentualer Anteil durch 1st Level Support gelöster Anfragen (> 80\%)
  \item Prozentualer Anteil durch 2nd Level Support gelöster Anfragen (< 20\%)
  \item Anzahl insgesamt gelöster Anfragen (< 5/Tag)
  \item Maximale Ausfalldauer von IT-Services (< 1h)
  \item Durchschnittliche Response Time von IT-Services (< 1s)
  \end{itemize}
\item Compliance
  \begin{itemize}
  \item Interner Zufriedenheitsfaktor (> 80\%)
  \item Häufigkeit von SLA-Reviews (1/Periode)
  \item Ersparnisse durch umgesetzte Maßnahmen (> 20\%)
  \end{itemize}
\end{itemize}

Für die Auswertung der Kennzahlen ist der CTO zuständig.  Seine
Aufgabe ist es ein Auge auf der Entwicklung im Vergleich zu vorigen
Perioden zu behalten, eingeschränkte Prognosen zu machen und Berichte
sowie Änderungsmaßnahmen mit dem Rest der Führungsetage zu teilen.
Als Ergänzung zu klassischen Reports sind Dashboards denkbar.  Der
dafür nötige Prozess kann also in die folgenden Schritte unterteilt
werden:

\begin{enumerate}
\item Planung der Ziele und Maßnahmen
\item Umsetzung dieser in die Praxis
\item Testen der Implementierung mit Analyse der Ergebnisse
\item Anpassung der Maßnahmen und/oder Ziele
\end{enumerate}

Sollte seine Analysen ergeben, dass die Resultate die selbstgesteckten Ziele
stark verfehlen, wird er nach Best Practices Ausschau halten und diese
wo es Sinn ergibt umsetzen.
